\documentclass[format=sigplan, review=false, screen=true]{acmart}
\usepackage[T1]{fontenc}
\usepackage[utf8]{inputenc}
\usepackage{fontspec}
\setmonofont[
  Contextuals={Alternate},
  Scale=0.9
]{Fira Code}
\makeatletter
\def\verbatim@nolig@list{}
\makeatother
\usepackage{microtype}
\linespread{0.97}

\usepackage{booktabs} % For formal tables
\usepackage{bussproofs}
\usepackage{graphics}
\usepackage{xcolor}

\usepackage{minted}
\usemintedstyle{friendly}
\setminted{
    frame=lines,
    framesep=2mm,
    baselinestretch=1.2,
    %bgcolor=lightgray,
    fontsize=\footnotesize,
    escapeinside=\#\#
    }

\usepackage{filecontents}

\usepackage[ruled]{algorithm2e} % For algorithms
\renewcommand{\algorithmcfname}{ALGORITHM}
\SetAlFnt{\small}
\SetAlCapFnt{\small}
\SetAlCapNameFnt{\small}
\SetAlCapHSkip{0pt}
\IncMargin{-\parindent}

\usepackage{listings}
\lstdefinestyle{mystyle}{
    %backgroundcolor=\color{backcolour},   
    commentstyle=\color{gray},
    keywordstyle=\color{blue},
    numberstyle=\tiny,
    stringstyle=\color{purple},
    basicstyle=\small\tt,
    breakatwhitespace=false,         
    breaklines=true,                 
    captionpos=b,                    
    keepspaces=true,                 
    %numbers=left,                    
    numbersep=5pt,                  
    showspaces=false,                
    showstringspaces=false,
    showtabs=false,
    frame=single,
    xleftmargin=1em,
    xrightmargin=1em,
    frame=shadowbox,
    rulesepcolor=\color{gray},
    tabsize=2
}
 
\lstset{style=mystyle,
  literate=
  {->} {$\to$} 2
  {<-} {$\leftarrow$} 2
  {=>} {$\Rightarrow$} 2
  {forall} {$\forall$} 1
  {exists} {$\exists$} 1
  {phi} {$\varphi$} 1
  {rho} {$\rho$} 1
  {kappa} {$\kappa$} 1
  {$nu$} {$\nu$} 1
  {$mu$} {$\mu$} 1
  {gamma} {$\gamma$} 1
  {subsetX} {$\subset$} 1
  {~>} {$\rightsquigarrow$} 2
  {<~>} {$\leftrightsquigarrow$} 3
  {elem} {$\in$} 1
}

\usepackage{cleveref}

% Metadata Information
\acmJournal{TWEB}
\acmVolume{9}
\acmNumber{4}
\acmArticle{39}
\acmYear{2018}
\acmMonth{3}
\copyrightyear{2018}
%\acmArticleSeq{9}

% Copyright
%\setcopyright{acmcopyright}
\setcopyright{acmlicensed}
%\setcopyright{rightsretained}
%\setcopyright{usgov}
%\setcopyright{usgovmixed}
%\setcopyright{cagov}
%\setcopyright{cagovmixed}

% DOI
\acmDOI{0000001.0000001}

% Paper history
\received{June 2018}
%\received[revised]{March 2009}
%\received[accepted]{June 2009}

% Document starts
\begin{document}
% Title portion. Note the short title for running heads
\title[Ghosts of Departed Proofs]{Functional Pearl: Ghosts of Departed Proofs}

\author{Matt Noonan}
\orcid{1234-5678-9012-3456}
\affiliation{%
  \institution{Kataskeue LLC, Input Output HK}
%  \streetaddress{Esty St}
  \city{Ithaca}
  \state{NY}
  \postcode{14850}
  \country{USA}}
\email{mnoonan@kataskeue.com}


\begin{abstract}

  Library authors often are faced with a design choice: should a function with
  preconditions be implemented as a partial function, or by returning a failure
  condition on incorrect use? Neither option is ideal. Partial functions lead
  to frustrating run-time errors. Failure conditions must be checked
  at the use-site,
  placing an unfair tax on the users who have ensured that the function's
  preconditions were correctly met.
  
  In this paper, we introduce an API design concept called ``ghosts of departed
  proofs'' based on the following observation: sophisticated preconditions can be
  encoded in Haskell's type system with no run-time overhead, by using proofs
  that inhabit phantom type parameters attached to \texttt{newtype} wrappers.
  The user expresses correctness arguments by constructing proofs to inhabit
  these phantom types.
  Critically, this technique allows the
  library \emph{user} to decide when and how to validate that the API's preconditions
  are met.

  The ``ghosts of departed proofs'' approach to API design can achieve many of the benefits
  of dependent types and refinement types, yet only requires some minor and well-understood
  extensions to Haskell 2010. We demonstrate the utility of this approach
  through a series of case studies, showing how to enforce novel invariants for lists,
  maps, graphs, shared memory regions, and more.
\end{abstract}


%
% The code below should be generated by the tool at
% http://dl.acm.org/ccs.cfm
% Please copy and paste the code instead of the example below.
%
 \begin{CCSXML}
<ccs2012>
<concept>
<concept_id>10011007.10011074.10011099.10011692</concept_id>
<concept_desc>Software and its engineering~Formal software verification</concept_desc>
<concept_significance>500</concept_significance>
</concept>
<concept>
<concept_id>10011007.10011006.10011008.10011024.10011025</concept_id>
<concept_desc>Software and its engineering~Polymorphism</concept_desc>
<concept_significance>300</concept_significance>
</concept>
<concept>
<concept_id>10011007.10011074.10011075.10011078</concept_id>
<concept_desc>Software and its engineering~Software design tradeoffs</concept_desc>
<concept_significance>100</concept_significance>
</concept>
<concept>
<concept_id>10003752.10003790.10002990</concept_id>
<concept_desc>Theory of computation~Logic and verification</concept_desc>
<concept_significance>300</concept_significance>
</concept>
</ccs2012>
\end{CCSXML}

\ccsdesc[500]{Software and its engineering~Formal software verification}
\ccsdesc[300]{Software and its engineering~Polymorphism}
\ccsdesc[100]{Software and its engineering~Software design tradeoffs}
\ccsdesc[300]{Theory of computation~Logic and verification}
%
% End generated code
%


\keywords{API design, software engineering, formal methods, higher-rank types}

\maketitle

% The default list of authors is too long for headers.
\renewcommand{\shortauthors}{M. Noonan}

%\input{samplebody-journals}

\section{Introduction}
\begin{quote}
  [Rico Mariani] admonished us to think about how we can build platforms that lead developers to write great, high performance code such that developers just fall into doing the ``right thing''. That concept really resonated with me. It is the key point of good API design. We should build APIs that steer and point developers in the right direction.
  
  \hfill --- Brad Abrams \cite{pitofsuccess}
\end{quote}

What is the purpose of a powerful type system? One practical perspective is
that a type system provides a mechanism for enforcing program
invariants at compile time. The desire to encode increasingly
sophisticated program invariants has led to a vast expanse of research
on more complex type systems, including dependent types \cite{augustsson1998cayenne,bove2009dependent}, refinement types \cite{freeman1991refinement}, linear
types \cite{wadler1990linear}, and many more. But despite this menagerie of powerful
type systems, workaday Haskell programmers have already been able to encode
surprisingly sophisticated invariants using nothing more than a
few well-understood extensions to the Damas-Hindley-Milner type system.

An early success story is the \texttt{ST} monad, which allows pure
computations to make use of local, mutable state. A phantom type parameter
and a clever use of rank-2 types in the \texttt{ST} monad's API gives
a compile-time guarantee that the local mutable state is invisible from the outside,
and hence the resulting computation really \emph{is} pure. As we will see, this
trick is just the tip of a rather large iceberg.

In this paper, we will take the perspective of a library author, writing in
Haskell 2010 (plus a few battle-tested language extensions). As a library
author, our goal will be to design \emph{safe} APIs that are also \emph{ergonomic}
for the end user. ``Safe'' means that we want to prevent the user from causing a
run-time error. ``Ergonomic'' means that we do not want the correct use of our
API to put an undue burden on the user.

\begin{filecontents*}{idioms.hs}
-- Unsafe API using non-total functions.
head :: [a] -> a
head xs = case xs of
  (x:_) -> x
  []    -> error "empty list!"

endpts = do
  putStrLn "Enter a non-empty list of integers:"  
  xs <- readLn
  if xs /= [] then return (head xs, head $ reverse xs)
              else endpts
----------------------------------------------------------
-- Returning Maybe / Optional values. Safe, but requires
-- the caller to pattern-match on the Maybe at every use,
-- even when the list is known to be non-empty. Frustrated
-- users cannot be blamed for using `fromJust`!
headMay :: [a] -> Maybe a
headMay xs = case xs of
  (x:_) -> Just x
  []    -> Nothing

safeEndpts = do
  putStrLn "Enter a non-empty list of integers:"  
  xs <- readLn
  case headMay xs of
    Just x -> return (x, fromJust (headMay $ reverse xs))
    _      -> safeEndpts
----------------------------------------------------------
-- "Ghosts of Departed Proofs". Safe. Does not return
-- an optional value; preconditions are checked early
-- and carried by "ghosts" (specialized phantom types).
rev_cons :: IsCons xs -> Proof (IsCons (Reverse xs))

gdpHead :: ([a] ~~ xs ::: IsCons xs) -> a
gdpHead xs = head (the xs) -- safe!

gdpEndpts = do
  putStrLn "Enter a non-empty list of integers:"  
  xs <- readLn
  name xs $ \xs -> case classify xs of
    IsCons evidence -> let ok = evidence >>= rev_cons in
      return (gdpHead xs, gdpHead (gdpRev xs ...ok))
    IsNil  evidence -> gdpEndpts
\end{filecontents*}

\begin{figure}
  \inputminted{haskell}{idioms.hs}
  \caption{Idioms for implementing the
    \texttt{head} function, along with usage examples.
    The \texttt{gdpHead} function can only be invoked in
    a context where the user has already proven that the
    list is non-empty, combining the simplicity of the
    first example with the safety of the second. \texttt{rev\_cons}
    is a proof combinator exported by the library to help the
    user prove that the reverse of a non-empty list is also non-empty. See
  \cref{full-gdp} for details.}
\end{figure}

\subsection{Common idioms for handling pre-conditions}

No matter the language, a programmer often has to write functions
that place constraints on their input. For example, the venerable
\texttt{head} function will extract the first element of a list,
but asks its users to only give it a non-empty list to operate on.
Now put yourself in the shoes of \texttt{head}'s author: how can
you ensure that \texttt{head} will be used properly? Let us recount
a variety of strategies used in the wild.

\paragraph{Run-time failure on bad inputs.}
Often the simplest approach is
  to have a function just fail on malformed inputs. The failure mode can
  be an immediate run-time error (as in Haskell's \texttt{head}), or result in
  undefined behavior (as in \texttt{C++}'s \texttt{std::vector<T>::front()}).
  
\paragraph{Returning a dummy value.}
  To avoid run-time errors, some APIs may have a ``dummy value''
  for indicating the result of a failed operation. For example, Common Lisp's
  \texttt{car} and \texttt{golang}'s \texttt{Front()} both return \texttt{nil}
  when passed an empty list. The caller must explicitly check for this dummy
  value. Other contortions may be needed if the container is also allowed to
  hold \texttt{nil}, to disambiguate between ``the input list is empty'' and
  ``\texttt{nil} is the first element of this list''.

\paragraph{Returning a value with an option type.}
  A similar ``dummy value'' strategy
  for languages with stricter typing discipline is to use an ``option type,'' such
  as Haskell's \texttt{Maybe} or Scala's \texttt{Option}. A value of type \texttt{Maybe T}
  cannot be used where a value of type \texttt{T} was expected, so the user must
  explicitly pattern match on the optional value to extract the result and handle the
  error case. This approach may lead to frustration when the user believes that the
  error case is not possible.
  
\paragraph{Modifying input types to exclude bad inputs.}
Finally, the API designer may select more restrictive types for the inputs in order
to make the function total. For example, some Haskell libraries make use of the
\texttt{NonEmpty} type for lists that contain at least one element. The \texttt{head}
function then becomes total. The user 
can prove that their list is non-empty by making use of the smart constructor
\texttt{nonEmpty :: [a] -> Maybe (NonEmpty a)}.\\
The drawbacks include duplication
(do we re-implement \texttt{length} for \texttt{NonEmpty}?)
and awkwardness when encoding preconditions that relate
several inputs (\textit{e.g.} requiring two lists to have the same length).

\subsection{Leading the user into temptation}
The ``return-an-optional-value'' idiom is well-known and popular in the functional
programming world. The author of a library function that returns \texttt{Maybe a}
can certainly sleep well at night, content in the knowledge that their function
will never cause a run-time error.

But what about the \emph{users} of that library? Has the library author helped the user stay
on a virtuous path, or have they led the user into temptation?

In fact, the author of the library has merely pushed extra responsiblity onto the user.
Every time the user applies a function that uses the optional-return idiom, they are obliged
to test the return value and handle the error case. Even worse, the user is still asked
to handle the error case when they have \emph{correctly} ensured that the function's
preconditions have been met! The library author sleeps well, while even the most vigilant
users are forced to toil against those impossible error cases.

No wonder so many well-meaning users reach for unsafe functions like \texttt{fromJust}!
They have already proved (to their own satisfaction) that the function is being used properly, so they rightly
feel justified in ignoring the error case entirely. But now we see how the user has been led into
a pit of despair: they have ended up with a program that is exactly as fragile as one where the library
author had used the run-time-failure idiom!\footnote{In fact, things are slightly \emph{worse}: we have also introduced a little
bit of extra allocation and indirection for creating and unpacking the return value in the non-error case.}
Even if the user has mentally constructed a proof that this specific use of \texttt{fromJust} is safe \emph{now},
who can say what will
happen as the software changes over time? Without tooling to ensure that the user's proof \emph{remains} valid,
the software is left in a brittle state.


For example, a recent snapshot of \texttt{hackage}
reveals over 2000 instances where the partial function
\texttt{fromJust} is applied to the
result of \texttt{Data.Map}'s \texttt{lookup}. Each one
of these instances is a vignette of a programmer
falling into a pit of despair:
they have a mental proof that a certain key must be
present in the map, but possess no mechanism to
\emph{communicating} that proof to the
\texttt{lookup} function. In frustration, they make the
pragmatic---but unsafe---decision to introduce partiality.

\subsection{Who is to blame?}
It would be easy to lay the blame at the foot of the the user.
After all, they were the ones who brought in partial functions!
But this perspective misses the point: when we return a \texttt{Maybe},
even a perfect user who has done their due diligence will be forced to handle an error case---exactly the
error case that they were so careful to avoid! The real problem is
that the conversion from a partial function to a \texttt{Maybe}-returning function is
a bit of a cheat on the part of the library author. Instead of
adding \texttt{Nothing} to a function's codomain, why not simply
restrict the function's domain to the set of valid inputs?
The user would still be responsible for ensuring that the inputs are valid but,
having done so, they would not be asked to introduce a spurious error handler.


\subsection{An alternative: Ghosts of Departed Proofs}
In the following sections, we will elaborate a design concept for
creating libraries that supports a dialogue between
 library and  user: the library can require that certain conditions
are met, and the user can explain how they have met those obligations.
The key features of this approach are as follows:
\paragraph{Properties and proofs are represented in code.}
  Proofs are concrete entities within the host language, and can
  be analyzed or audited independently. In the tradition of
  the Curry-Howard correspondence, propositions are represented
  by types, and the proof of a proposition will be a value of that type.
\paragraph{Proofs carried by phantom type parameters.}
  To ensure that proof-carrying code does not have a run-time cost, proofs will only
  be used to inhabit types that appear as \emph{phantom type variables} attached to
  \texttt{newtype} wrappers.
  The \texttt{newtype} wrapper is erased during compilation, leaving no run-time cost and
  no evidence of these proofs in the final executable.
  The phantom type parameter is only used as a mechanism for transmitting the
  ``ghost of a departed proof'' to the library API.
  The name ``ghost proof'' is meant to suggest the related concept of \emph{ghost variables} in software
  verification \cite{leavens1999jml}, and to emphasize the idea that the proof is non-corporeal: no
  artifacts related to the proof should ever be discernible from the compiler's output.
\paragraph{Library-controlled APIs to create proofs.}
  Library authors should retain control over how domain-relevant proofs can be created.
  That is, the library author should be the only one able to introduce new axioms about
  the behavior of their API.
  This may mean exporting functions that create values with known properties, or that
  classify a value into mutually disjoint refinements,
%%subtypes (\texttt{classify} in \cref{size-evidence})
or that introduce existentially-quantified properties (\texttt{name} in \Cref{name-module}, or \texttt{withMap}
in \Cref{justified-api}).
\paragraph{Combinators for manipulating ghost proofs.}
  Libraries may export a selection of combinators so that the user can
  mix and match the evidence at hand to produce a satisfactory proof of a
  safety property. The goal is to enrich the vocabulary of the user, so
  that they can productively communicate their proofs to the library.


\subsection{The structure of this paper}
In this paper, we will use a series of case studies to show how library authors can use
ghosts of departed proofs (GDP) to create
APIs that are both \emph{safe} and \emph{ergonomic}. By \emph{safe}, we mean that the
user cannot cause a run-time error or undefined behavior when using the API. Incorrect uses
will become compile-time errors. By \emph{ergonomic}, we mean that the API is straightforward
enough that the user is not tempted to subvert the library's safety guarantees by using
unsafe functions. Crucially, we want the user to be able to \emph{communicate} their
informal proofs to the library. If the user believes that a precondition has been met,
they should be able to explain \emph{why} to the library!

The GDP design concept is relatively simple to implement. Each case study includes example library code,
along with usage examples. The examples in this paper are self-contained, and are bundled together in
a project suitable for further experimentation \cite{this}.


\section{Case Study \#1: Sorted lists}

It is almost inevitable that a programmer will, at some point, be asked to work
with lists that have been sorted in one way or another. To ensure correctness,
the programmer may need to carefully manage various invariants, such
as ``all of these lists must have been sorted by the same comparator''. For a concrete
example, consider these \texttt{sortBy} and \texttt{mergeBy} functions:
\begin{minted}[frame=none]{haskell}
sortBy  :: (a -> a -> Ordering) -> [a] -> [a]

-- Usage constraint: in `mergeBy comp xs ys`, the
-- input lists `xs` and `ys` should also be sorted
-- by the same comparator `comp`.
mergeBy :: (a -> a -> Ordering) -> [a] -> [a] -> [a]
mergeBy comp xs ys = go xs ys
  where
    go []  ys' = ys'
    go xs' []  = xs'
    go (x:xs') (y:ys') = case comp x y of
      GT -> y : go (x:xs') ys'
      _  -> x : go xs' (y:ys')
\end{minted}
This efficient $O(n+m)$ implementation of \texttt{mergeBy} is easy to write,
but it comes with a hidden cost to the end user. Anybody who uses \texttt{mergeBy}
must ensure that the two input lists have been sorted by the same comparator.
If the user accidentally fails to sort the two inputs, or does not sort them in the same way,
\texttt{mergeBy} will quietly produce nonsense and introduce a subtle bug.

It would be possible to implement a version of \texttt{mergeBy} that
carefully inspected the inputs \texttt{xs} and \texttt{ys} as it
proceeded, and only produced a result if the inputs met the sorting
requirement. But this would impose a runtime cost on every use of
\texttt{mergeBy}, increase the complexity of its implementation,
and change the result type to \texttt{Maybe [a]}. And then what?
Most users of \texttt{mergeBy} would argue to themselves ``This is
absurd! I already know that I sorted the input lists properly. This
function will never result in \texttt{Nothing}.'' It would be hard
to blame the user when they reach for an  unsafe function like
\texttt{fromJust}.

Clearly, everybody loses out in the above scenario. The
library author is inconvenienced by the increased implementation complexity.
The user is inconvenienced by the decreased performance and the need to
pattern match on the result, even when they  already know the
outcome of that match. No wonder that the status quo is to prominently display
a stern warning in the documentation, admonishing
any user who tries to \texttt{mergeBy} what they didn't \texttt{sortBy}.

But what if the user really \emph{does} have proof that the input lists have
been sorted properly? Can we devise a mechanism that allows the user to communicate
this proof to \texttt{mergeBy}?

\subsection{Conjuring a name}

\begin{filecontents*}{named.hs}
module Named
  (Named, type (~~), name, Defn, defn, Defining) where

import The
import Data.Coerce

newtype Named name a = Named a
instance The (Named name a) a
type a ~~ name = Named name a

-- Morally, the type of `name` is
--      a -> (exists name. (a ~~ name))
name :: a -> (forall name. (a ~~ name) -> t) -> t
name x k = k (coerce x)

data Defn = Defn
type Defining f = (Coercible f Defn, Coercible Defn f)

-- Allow library authors to introduce their own names.
defn :: Defining f => a -> (a ~~ f)
defn = coerce
\end{filecontents*}
\begin{figure}[b]
  \inputminted{haskell}{named.hs}
  \caption{A module for attaching ghostly names to values. The rank-2 type of \texttt{name},
    making use of a polymorphic continuation, is one way to emulate an existential type in
    Haskell. The \texttt{Defining} typeclass
    leverages \texttt{GHC}'s magic \texttt{Coercible} class \cite{Breitner:2014:SZC:2692915.2628141},
    ensuring that \texttt{defn} is available a library's author but not
    to the library's users. \Cref{lemma-demo} provides a usage example.\label{name-module}}
\end{figure}


The first challenge is how to express the idea of two  comparators
being ``the same''. In a language that supports equality tests on functions,
you could imagine a solution where the \texttt{sortBy} function returns both the sorted
list and a reference to the comparator that was used; \texttt{mergeBy} could
then check that the comparators matched. But this has a run-time cost for carrying
around the comparator references, and it still would require \texttt{mergeBy} to
return \texttt{Nothing} if it was given bogus arguments.

A different solution, in line with the GDP concept, is to introduce a \texttt{newtype} wrapper equipped with
a phantom type parameter \texttt{name}.
In code, we will write this wrapper as \verb|a ~~ n|, to be read as
``values of type \texttt{a} with name \texttt{n}''. To ensure that there is no
run-time penalty for using names, \verb|a ~~ n| is implemented as a \texttt{newtype}
around \texttt{a}, with a phantom type parameter \texttt{n}. A simple module for named values
can be found in \Cref{name-module}; the key feature is the exported \texttt{name}
function that expresses the concept ``any value can be given a name''.

Since Haskell does not support existentially-quantified types, we have to jump through
a small hoop with \texttt{name}. Instead of directly returning a value with a name attached,
\texttt{name} says to the user ``tell me what you wanted to do with that named value,
and I'll do it for you''. This slight-of-hand is responsible for the rank-2
signature of \texttt{name}. The user must hand \texttt{name} a computation that is
entirely agnostic about the name that will be chosen. More on this point in \cref{ghost-danger}.
%In \cref{case4}, we will extend this module with extra functionality in support of
%custom, library-defined names.

Once we have introduced names, it becomes handy to have a uniform way of stripping names
and other phantom data from a value. We do this with a simple two-parameter typeclass,
like so:
\begin{minted}[frame=none]{haskell}
class The d a | d -> a where
    the :: d -> a
    default the :: Coercible d a => d -> a
    the = coerce
\end{minted}
By using this default signature for \texttt{the}, most instances of \texttt{The}
can be declared with an empty body:
\begin{minted}[frame=none]{haskell}
instance The (a ~~ name) a
\end{minted}
The default method's use of a safe coercion helps ensure that removing the name from a value
incurs no run-time cost.

\begin{filecontents*}{ordered.hs}
module Sorted (Named, SortedBy, sortBy, mergeBy) where   

import The
import Named

import           Data.Coerce
import qualified Data.List       as L
import qualified Data.List.Utils as U

newtype SortedBy comp a = SortedBy a
instance The (SortedBy comp a) a
  
sortBy :: ((a -> a -> Ordering) ~~ comp)
       -> [a]
       -> SortedBy comp [a]
sortBy comp xs = coerce (L.sortBy (the comp) xs)

mergeBy :: ((a -> a -> Ordering) ~~ comp)
        -> SortedBy comp [a]
        -> SortedBy comp [a]
        -> SortedBy comp [a]
mergeBy comp xs ys =
  coerce (U.mergeBy (the comp) (the xs) (the ys))        
\end{filecontents*}

\begin{filecontents*}{usageO.hs}
import Sorted
import Named
main = do
  xs <- readLn :: IO Int
  ys <- readLn
  name (comparing Down) $ \gt -> do
    let xs' = sortBy gt xs
        ys' = sortBy gt ys
    print (the (mergeBy gt xs' ys'))
\end{filecontents*}

\begin{figure}
  \inputminted{haskell}{ordered.hs}
  \caption{A module for working with lists that have been sorted by an arbitrary
    comparator. The refinement \texttt{SortedBy comp} is used to denote values that
    have been sorted by the comparator named \texttt{comp}.\label{sorted-module}}
\end{figure}


\begin{figure}
  \inputminted{haskell}{usageO.hs}
  \caption{Using the module developed in \Cref{sorted-module}.\label{sorted-module-demo}}
  \end{figure}

\subsection{Implementing a safe API for sorting and merging}
Now that we know how to attach ghostly names to values, we can tackle the design of a
safe and ergonomic interface to \texttt{mergeBy}. In \Cref{sorted-module}, we begin by defining
a \texttt{newtype} wrapper \texttt{SortedBy comp} to that represents the predicate
``$x$ has been sorted by the comparator named \texttt{comp}''. The
wrapper's meaning is imbued by the type of \texttt{sortBy}, which takes a \emph{named}
comparator and a list, and produces a list that has been \texttt{SortedBy comp}.
Note that by \emph{not} exporting \texttt{SortedBy}'s constructor, we have ensured that
the \emph{only} way to obtain a value of type \texttt{SortedBy comp [a]} is through the
\texttt{sortBy} or \texttt{mergeBy} functions. The user is not allowed to assert that a
list is \texttt{SortedBy comp} by fiat.

The implementation is straightforward enough: we just coerce away the name of the comparator,
apply the simpler version of \texttt{sortBy} from \texttt{Data.List}, and then
introduce the \texttt{SortedBy comp} predicate by coercing the result. Since the coercions have
no run-time effect, the code generated by the compiler for our GDP-style \texttt{sortBy} is
simply a call to \texttt{Data.List}'s \texttt{sortBy}!

Similarly, the generated code for our \texttt{mergeBy} will just call the ``normal'' \texttt{mergeBy}.
But notice the argument types of the GDP-style \texttt{mergeBy} in \Cref{sorted-module}. The user
must hand \texttt{mergeBy} a named comparator, plus two lists that have been sorted by that very same
comparator. No stern warnings in the documentation are required: if the user tries to \texttt{mergeBy}
what they didn't \texttt{sortBy}, the program will simply fail to compile!

We have successfully developed a safe API for \texttt{sortBy} and \texttt{mergeBy}, but how ergonomic
is it? A usage example appears in \Cref{sorted-module-demo}. The program is almost identical to one
that uses the standard versions of \texttt{sortBy} and \texttt{mergeBy}, except for the line where
we attach a ghostly name to \texttt{comparing Down}. We are asking very little more from the user,
yet end up with an API that cannot be used incorrectly.

\subsection{Applications to user code}
Although the library author retains control over the \emph{introduction} of ghost proofs, the
user is still able to leverage these proofs for their own purposes, beyond the library author's original
design. For example, the user can write a simple function that extracts the minimal element of
a list with respect to a given comparator:
\begin{minted}[frame=none]{haskell}
minimum_O1 :: SortedBy comp [a] -> Maybe a
minimum_O1 xs = case the xs of
    []    -> Nothing
    (x:_) -> Just x
\end{minted}
Thanks to the meaning given to \texttt{SortedBy comp} by the \texttt{Sorted} API, this user-defined
function offers a strong guarantee that it can only be called on a sorted list. Despite being user-defined,
this function cannot be used incorrectly. Did you forget to sort the list before calling \texttt{minimum\_O1}?
Then your program will not compile.

\subsection{Aside: On the danger of naming a ghost}\label{ghost-danger}

Let us return for a moment to the somewhat unusual type of \texttt{name} in \Cref{name-module}.
Is all of this business about rank-2 types \emph{really} necessary, or is it merely ivory tower
bloviation?  You may well wonder, why not just have a function with a simple type like this:

\begin{minted}[frame=none]{haskell}
any_name :: a -> (a ~~ name)
any_name = coerce
\end{minted}

At its core, the question is really about \emph{who gets to choose} what \texttt{name} will be.
In the signature of \texttt{any\_name}, the \emph{caller} gets to select the types \texttt{a}
and \texttt{name}. In particular, they can attach any name they would like!
If that still does not sound so bad, consider this code:

\begin{minted}[frame=none]{haskell}
data Simon
  
up, down :: (Int -> Int -> Ordering) ~~ Simon
up   = any_name compare
down = any_name (comparing Down)

list1 = sortBy up   [1,2,3]
list2 = sortBy down [1,2,3]

merged = the (mergeBy up list1 list2) :: [Int]
-- [1,2,3,3,2,1]
\end{minted}
\noindent
The user has decided to name two different functions \texttt{Simon}, subverting the
guarantees offered by the API of the \texttt{Sorted} module. It is dangerous to
name a ghost!

Now compare this to the analogous program, using \texttt{name} instead of \texttt{any\_name}:
\begin{minted}[frame=none]{haskell}
name compare $ \up ->
  name (comparing Down) $ \down ->
    let list1 = sortBy up   [1,2,3]
        list2 = sortBy down [1,2,3]
    in the (mergeBy up list1 list2)
\end{minted}
\noindent
Attempting to compile this program results in a type error:

\begin{lstlisting}
  • Couldn't match type "name1" with "name"
        ...
    Expected type: SortedBy name [Integer]
      Actual type: SortedBy name1 [Integer]
\end{lstlisting}
\noindent
What is the critical difference between these two examples? In the first, a user is
allowed to \emph{create} a named value by fiat. In the second, the user is only allowed to \emph{consume} a named value, by
providing a polymorphic function that can work with \emph{any} named value. The library's API provides
a helper function---in this case, \texttt{name}---for applying the consumer to a normal, unnamed value.
In practice, it is as if the
library has a secret supply of names, and selects one to use in a manner that is not
predictable (or even inspectable!) to the user.

%A general rule of thumb for library authors is:
%\emph{a ghost should not appear in the return type,  unless it also appears in an argument's type}. This simple rule ensures that
%the user of the library will not be allowed to materialize ghosts out of thin air.

\section{Case Study \#2: Sharing state threads}
The trick for using rank-2 types to conjure names outside of the user's control was
inspired by the \texttt{ST} monad and its rank-2 \texttt{runST :: (forall s. ST s a) -> a}
function \cite{launchbury1994lazy}. In this brief case-study, we elaborate the connection
between the \texttt{ST} monad and GDP-style names. The new perspective suggests novel
extensions to the \texttt{ST} API.
In \Cref{st-api} we recall the basic \texttt{ST} API \cite{launchbury1994lazy}, writing \texttt{St} to
disambiguate our version from the existing type in \texttt{Control.Monad.ST}.

In their safety analysis of the \texttt{ST} monad, Timany \textit{et al.} proposed thinking of the \texttt{s} parameter as
representing a name attached to a region of the heap \cite{timany2017logical}.
Informally, we can think of \texttt{ST s a} as representing a \texttt{State} monad over
\emph{named regions}, like so:
\begin{minted}[frame=none]{haskell}
data Region = Region
type St s a = State (Region ~~ s) a

runSt :: (forall s. St s a) -> a
runSt action = name Region (evalState action)
\end{minted}

The notion of treating the \texttt{ST} monad's phantom type as a region name immediately leads to ideas for
other primitives. Once we can name regions, why not go on to invent more detailed names to describe
the minute contours of those regions? For example, let us add a binary type constructor $\cap$ so that \texttt{s $\cap$ s'}
names the region at the intersection of \texttt{s} and \texttt{s'}. We are quickly led to an API similar to \Cref{st-sharing-api} that
supports a new capability: individual sub-computations, at their discretion, may decide to share mutable reference cells with other sub-computations.

\begin{filecontents*}{st1.hs}
runSt    :: (forall s. St s a) -> a

newRef   :: a -> St s (a #$\in$# s)
readRef  :: (a #$\in$# s) -> St s a
writeRef :: (a #$\in$# s) -> a -> St s ()
\end{filecontents*}

\begin{filecontents*}{st2.hs}
runSt2 :: (forall s s'. St (s #$\cap$# s') a) -> a

liftL :: St s a -> St (s #$\cap$# s') a
liftR :: St s' a -> St (s #$\cap$# s') a

share :: (a #$\in$# s) -> St s (a #$\in$# (s #$\cap$# s'))

use  :: (a #$\in$# (s #$\cap$# s')) -> (a #$\in$# s)
symm :: (a #$\in$# (s #$\cap$# s')) -> (a #$\in$# (s' #$\cap$# s))
\end{filecontents*}

\begin{figure}
  \inputminted{haskell}{st1.hs}
  \caption{The standard ``state thread'' API. We write \texttt{a $\in$ s} to
    denote a reference cell of type \texttt{a} in the memory region named \texttt{s}.
    In \texttt{Control.Monad.ST}, we would write \texttt{a $\in$ s} as
    \texttt{STRef s a}.\label{st-api}}
\end{figure}

\begin{figure}
  \inputminted{haskell}{st2.hs}
  \caption{Extending the state thread API with shared references.\label{st-sharing-api}}
\end{figure}
In effect, \texttt{runSt2} lets the user run a computation that makes use of
two partially-overlapping memory regions. Within that computation, the user
can run sub-computations bound to one or the other memory region. Furthermore,
a sub-computation can move any variable that it owns into the common overlap
via \texttt{share}. An example is shown in \Cref{st-example}, where one sub-computation
creates two cells: one private, and the other shared. A second sub-computation has unconstrained
access to the shared cell. Yet even though the private reference is also in scope during
the second sub-computation, any attempts to access it will fail to compile.

\begin{filecontents*}{st.hs}
stSharingDemo :: Bool
stSharingDemo = runSt2 $ do
  -- In the "left" memory region, create and return
  -- two references; one shared, and one not shared.
  (secret, ref) <- liftL $ do
      unshared <- newRef 42
      shared   <- share =<< newRef 17
      return (unshared, shared)
  -- In the "right" memory region, mutate the shared
  -- reference. If we attempt to access the non-shared
  -- reference here, the program will not compile.
  liftR $ do
      let mine = use (symm ref)
      x <- readRef mine
      writeRef mine (x + 1)
  -- Back in the "left" memory region, verify that the
  -- unshared reference still holds its original value.
  liftL $ do
      check <- readRef secret
      return (check == 42)
\end{filecontents*}

\begin{figure}
  \inputminted{haskell}{st.hs}
  \caption{An \texttt{ST}-style pure computation using local mutable
    references. By introducing two named regions at once, we can extend
    the \texttt{ST} API with new capabilities such as shared references.
    Although the \texttt{secret} reference is in scope during the calculation
    in the ``right'' region, any attempted access will fail to compile.\label{st-example}}
\end{figure}

\section{Case Study \#3: Key-value lookups}

In this section, we will perform a \textit{post mortem} on those
2000 moments, forever enshrined on Hackage, where a programmer fell into
the pit of despair and followed a map lookup by \texttt{fromJust}.

It is instructive to compare the two functions \texttt{lookup :: k -> Map k v -> Maybe v}
and \texttt{lookup :: (k $\in$ ks) -> JMap ks k v -> v}. We do not intend to claim that
one of these is better than the other. Instead, the claim is that these two functions
\emph{reflect different states of the user's knowledge}.

If the user legitimately does not know whether or not a key is present, then the
\texttt{Maybe}-returning \texttt{lookup} is called for. The user's incomplete knowledge
about the result of the operation is exactly reflected in the return type, so they will
not feel inconvenienced by the need to handle both the \texttt{Just v} (key present)
and \texttt{Nothing} (key absent) cases.

On the other hand, if the user already believes the key should be present based on some
external evidence, then they will be happier writing a program that does not need to handle
the impossible missing-key state. But they must demonstrate that evidence to the library
somehow!  ****

\subsection{Application: well-formed adjacency lists}

The power of this method becomes more apparent when considering maps where
the values are expected to reference the keys in some way. For example, a
simple representation for directed graphs with vertex type \texttt{v} is:
\begin{minted}[frame=none]{haskell}
type Digraph v = Map v [v]
\end{minted}
mapping each vertex to its list of immediate neighbors. Well-formed \texttt{Digraph}s
should satisfy the property that every vertex referenced in any neighbor list is also
a valid key in the adjacency map.

Traditionally, graph APIs using adjacency representations require well-formed
graphs, but make it the user's responsiblity to ensure well-formedness. For example,
the \texttt{Data.Graph} API from \texttt{containers} has a graph constructor that
will silently discard edges with targets that do not appear in the node list.
%%In \texttt{C++}, \texttt{boost::adjacency\_list} has an \texttt{add\_edge} function
%%that returns a \texttt{bool}. This value must be

The GDP-style structure of the \texttt{justified-containers} API makes it easy to
translate the notion of ``well-formed adjacency list'' into an invariant that can
be checked by the compiler. We simply write what we mean: a well-formed adjacency
list is a map from vertices to a list of vertices that are keys of that same map.
In other words:
\begin{minted}[frame=none]{haskell}
type Digraph vs v = JMap vs v [Key vs v]
\end{minted}
With the help of this type, a user can now enforce the invariant ``this adjacency map
must be well-formed'' at compile time. A similar strategy can be used to eliminate a
whole class of bugs when using symbol tables, evaluation contexts,
database models, and or any other data structure based around a recursive key-value store.

\begin{filecontents*}{justified.hs}
newtype JMap ks k v = JMap (Map k v) deriving Functor
newtype Key  ks k   = Key k

instance The (JMap ks k v) (Map k v)
instance The (Key ks k) k

member   :: k -> JMap ks k v -> Maybe (Key ks k)
lookup   :: Key ks k -> JMap ks k v -> v
reinsert :: Key ks k -> v -> JMap ks k v -> JMap ks k v
withMap  :: Map k v -> (forall ks. JMap ks k v -> t) -> t
\end{filecontents*}

\begin{filecontents*}{justified-usage.hs}
test = Map.fromList [ (1, "Hello"), (2, "world!") ]

name test $ \table ->
  case member k table of
    Nothing  -> putStrLn "Missing key!"
    Just key -> do
      let table'  = reinsert key "Howdy" table
          table'' = fmap (map upper) table
      putStrLn ("Value in map 1: " ++ lookup key table)
      putStrLn ("Value in map 2: " ++ lookup key table')
      putStrLn ("Value in map 3: " ++ lookup key table'')
-- Output:
--   Value in map 1: Hello
--   Value in map 2: Howdy
--   Value in map 3: HELLO
\end{filecontents*}

\begin{figure}
  \inputminted{haskell}{justified.hs}
  \caption{A fragment of the API from \texttt{justified-containers}.
    The GDP-style predicates \texttt{k $\in$ ks} and \texttt{JMap ks k v} are used to represent
    ``a value of type \texttt{k} belonging to the set \texttt{ks}'' and ``a map with key set \texttt{ks}'',
    respectively. \label{justified-api}}
\end{figure}
\begin{figure}
  \inputminted{haskell}{justified-usage.hs}
  \caption{A usage example for the API in \Cref{justified-api}. The \texttt{member} function is used
    to check if a key is present in \texttt{table}; within the scope of the \texttt{Just} case, \texttt{key}
    carries a phantom proof of its presence in \texttt{table}. The same phantom proof can also be used as evidence that
    \texttt{key} is present certain other maps as well, such as \texttt{table'} (\texttt{table} with a
    value changed) and \texttt{table''} (\texttt{table} modified by \texttt{fmap}).} 
\end{figure}

%%%% Maybe not so relevant?
%% \subsection{Application: Faster lookup}
%% Containers such as \texttt{Data.Map} that make use of a total ordering on the keys
%% generally are designed to allow for lookup in $O(\log n)$ comparisons, where $n$ is the
%% total number of keys in the map. If the comparator used for the keys is unusually
%% expensive, ***. For this situation, \texttt{Data.Map} module offers ``indexed'' versions
%% of many operations, identifying keys by their order in the map rather than by their actual
%% value. Lookup with a key's index then requires $O(\log n)$ \emph{integer} comparisons.

%% Although \texttt{justified-containers} defines a simple \texttt{newtype} wrapper for
%% the key-plus-phantom-proof type, more interesting information about the location of
%% the key within the corresponding data structure can sometimes be attached.

%% For example, imagine a simple binary search tree backed by a vector of key-value pairs.
%% As in the previous example, we will give the \texttt{BST} type a phantom parameter that
%% represents the set of valid keys present in the tree. But instead of wrapping the key
%% type directly, we will use an index-plus-phantom-proof representation for keys.


\subsection{Changing the key set}\label{changing-keys}
But what about maps that are related, yet do not have exactly the same key sets?
As a concrete example, consider the \texttt{insert} function. Although \texttt{insert} will usually
modify the key set of a map, we still know quite a lot about the keys in the updated map.
 Imagine you were a user, in possession of a key and a proof
that it is present in the original map. It would be quite frustrating if  we were
unable to use that same key freely in the expanded map!
The library author, anticipating this need, should provide a proof combinator for
turning a proof that \texttt{k} is a valid key of \texttt{m} into a proof that
\texttt{k} is also a valid key of \texttt{insert k' v m} for any \texttt{k'}.

To support this use-case, \texttt{justified-containers} provides the functions
\texttt{inserting} and \texttt{deleting}, with these signatures:
\begin{minted}{haskell}
deleting  :: Ord k => Key ks k -> JMap ks k v
          -> (forall ks'. JMap ks' k v
                      -> (Key ks' k -> Key ks k)
                      -> t)
          -> t

inserting :: Ord k => k -> v -> JMap ks k v
          -> (forall ks'. JMap ks' k v
                       -> (Key ks k -> Key ks' k)
                       -> Key ks' k
                       -> t)
          -> t
\end{minted}
Since these functions each result in maps with new key sets, we must
introduce the ghost of these new key sets inside another \texttt{forall}.
But what are the other parameters for? In the case of \texttt{inserting},
the computation has access to:
\begin{enumerate}
\item The updated map, of type \texttt{JMap ks' k v}. The phantom type \texttt{ks'}
  represents the key set \texttt{ks}, updated with the newly-inserted key.
\item A function that represents the inclusion of \texttt{ks} into \texttt{ks'}.
  The user can apply this function to convert a proof that a certain key belonged to the
  old map (a value of type \texttt{Key ks k}) into a proof that the key also belongs to the new map (a value of type \texttt{Key ks' k}).
\item Evidence that the new key is present in the new key set.
\end{enumerate}

Similarly, the continuation for \texttt{deleting} has access to both the updated map
and a representation of the inclusion of keys from the new map into the old map.

The library author must do a bit of a balancing act here. It is important to provide
the user with a supply of evidence and proof combinators so that they are able to
express the fact that the API is being used properly, but it is not always clear how
much information should be added. For example, the user may well want to argue that a
every key \emph{other} than the deleted one is still present in a map modified by
\texttt{deleting}, but the API provides no straightforward way to do this.

\section{Case Study \#4: Arbitrary invariants}\label{full-gdp}

As a final case study, we investigate how the GDP technique can be used to
create APIs that can enforce arbitrary conditions on functions.

In the previous case studies, we saw how introducing names and predicates can
help us develop safe APIs that allow the user to express correctness proofs.
However, there are a few aspects that remained awkward.

First, we have several
ways that a name-like entity could be introduced: either via the \texttt{name} operator
itself, or through other library-defined rank-2 functions like \texttt{runSt2} or \texttt{withMap}.
It would be nice if the same mechanism could be used for all of these cases.

Second, we made extensive use of ghostly proofs carried by phantom type parameters. But these
phantom types needed something to attach to, so we introduced various domain-specific \texttt{newtype}
wrappers such as \texttt{SortedBy}, \texttt{$\in$}, and \texttt{JMap}. Each library exported its own
idiosyncratic proof combinators for working with its \texttt{newtype} wrappers. It would be better to have
a uniform mechanism for expressing, carrying, and manipulating these proofs.

In this case study, we will consider what kind of APIs we could write if we separated
type-level names for each value from the constraints we want to place on those values.
For example, let us return to the \texttt{head} function. We want ensure that the
user only calls \texttt{head} on a list \texttt{xs} with outer constructor \texttt{(:)} (``cons'').
To express this condition, we introduce one more \texttt{newtype} wrapper,
written \texttt{:::} and pronounced ``such that''. Altogether, the phrase 
\verb|(a ~~ n ::: p)| should be read ``a value of type \texttt{a}, named \texttt{n}, such
that condition \texttt{p} holds.''


We can now write, very explicitly, the requirement that the library places on the user
of \texttt{head}: the parameter, called \texttt{xs}, must have outermost constructor
\texttt{(:)}. So we simply introduce a predicate \texttt{IsCons}, and write down the definition:
\begin{minted}[frame=none]{haskell}
newtype IsNil  xs = IsNil  Defn
newtype IsCons xs = IsCons Defn

gdpHead :: ([a] ~~ xs ::: IsCons xs) -> a
gdpHead xs = case the xs of (x:_) -> x
\end{minted}
The \texttt{:::} type is similar to the \texttt{Refined} type from the \texttt{refinement} library \cite{refined},
but it gains extra power when used together with names.
In particular, names provide the mechanism that allows us to take predicates about specific values and
encode them at the type level.
The type \verb|([a] ~~ xs ::: IsCons xs)| becomes a statement about the particular list being passed to \texttt{head}.
The library user is now free to come up with a proof of \texttt{IsCons xs} in whatever way they please.


\begin{filecontents*}{ex1.hs}
-- API functions
gdpRev :: ([a] ~~ xs) -> ([a] ~~ Reverse xs)
gdpRev xs = defn (reverse (the xs))

length :: ([a] ~~ xs) -> (Integer ~~ Length xs)
length xs = defn (Prelude.length (the xs))

zipWith :: ((a -> b -> c) ~~ f)
         -> ([a] ~~ xs ::: Length xs == n)
         -> ([a] ~~ ys ::: Length ys == n)
         -> ([a] ~~ ZipWith f xs ys)
zipWith f xs ys =
  defn (Prelude.zipWith (the f) (the xs) (the ys))

-- Names for API functions
newtype Length  xs      = Length  Defn
newtype ZipWith f xs ys = ZipWith Defn
newtype Reverse xs      = Reverse Defn

-- Lemmas (all bodies are `axiom`)
rev_length :: Proof (Length (Reverse xs) == Length xs)
rev_rev    :: Proof (Reverse (Reverse xs) == xs)
rev_cons   :: IsCons xs -> Proof (IsCons (Reverse xs))

data ListCase xs = IsNil  (Proof (IsNil  xs))
                 | IsCons (Proof (IsCons xs)) 

classify :: ([a] ~~ xs) -> ListCase xs
classify xs = case the xs of
  []    -> IsNil  axiom
  (_:_) -> IsCons axiom
\end{filecontents*}

\begin{filecontents*}{ex2.hs}
dot :: ([Double] ~~ vec1 ::: Length vec1 == n)
    -> ([Double] ~~ vec2 ::: Length vec2 == n)
    -> Double
dot vec1 vec2 = name (*) $ \mul ->
  sum (the (zipWith mul vec1 vec2))

-- Compute the dot product of a list with its reverse.
dot_rev :: [Double] -> Double
dot_rev xs = name xs $ \vec ->
  dot (vec ...refl) (reverse vec ...rev_length)
\end{filecontents*}

\begin{figure}
    \inputminted{haskell}{ex1.hs}
    \caption{A GDP-style module for manipulating and reasoning about lists.
      A variety of lemmas are exported by the module, to provide the
      user with a rich set of building blocks for constructing safety proofs.
       \label{lemma-demo}}
\end{figure}
\begin{figure}
    \inputminted{haskell}{ex2.hs}
    \caption{A user-defined dot product function that can only be used on same-sized lists,
      and a usage example. In the implementation of \texttt{dot\_rev}, the user expresses a proof that
      \texttt{vec} and \texttt{reverse vec} have the same length to make the use of \texttt{dot}
       well-typed.\label{dot-product}}
\end{figure}

\subsection{An EDSL for ghostly proofs}

\begin{filecontents*}{tableaux.hs}
type Theorem p q = (p --> q) --> (Not q --> Not p)
proof1, proof2 :: Proof (Theorem p q)
  
proof1 =
  impl_intro $ \p_implies_q ->
    impl_intro $ \not_q ->
      not_intro $ \p -> do
        q <- impl_elim p_imples_q p
        contradicts q not_q

proof2 = tableaux
\end{filecontents*}

\begin{figure}
  \inputminted{haskell}{tableaux.hs}
  \caption{Proving the same theorem in two different ways. The first proof
    uses the \texttt{Monad} instance for
    \texttt{Proof} and the inference rules from \Cref{predicate-logic}. The second
    proof uses a typechecker plugin, exposed through the \texttt{tableaux}
    function (\Cref{tactics}).
    \label{tableaux-example}}
\end{figure}
We now have a mechanism for encoding arbitrary properties as phantom types. But how will the user
create ghostly proofs to inhabit those phantoms types?
We can begin with a very simple \texttt{Proof} type,
sporting a single phantom type and exactly one non-bottom value:
\begin{minted}[frame=none]{haskell}
data Proof p = QED deriving (Functor, Applicative, Monad)
\end{minted} 
From this humble beginning, we can encode all of the inference rules of natural deduction as functions that
produce terms of type \texttt{Proof p}.
\Cref{predicate-logic} gives a small taste of the basic syntax and encoded inference rules, along with
a collection of combinators for building up larger proofs from smaller steps.
The monad instance for \texttt{Proof} turns out to be very useful, enabling proofs to be written in
\texttt{do}-notation. To get a sense of the flavor, see \texttt{proof1} in \Cref{tableaux-example}.

Once we have constructed a proof of type \texttt{Proof p}, we can attach that proof to any value of type \texttt{a}
to get a value of type \verb|(a ::: p)|. Note that \texttt{p} will often be a proof \emph{about} the wrapped value,
but that is not required! Any value can carry any proof; the only thing that binds a value to a proof it is carrying
is the use of a name.

\subsection{Naming library functions}
A library author may want to export a lemma such as ``reversing a list twice gives the original list''.
To express this idea, it is not sufficient to have a name for ``the original list''. We must also be able to
name some of the library's functions. This observation finally reveals the motivation for  the \texttt{defn}
function in \Cref{name-module}: we want to let the library author introduce new names for their own functions,
so they will be able to export \texttt{Proof}s representing properties that those functions should satisfy.

In the case of the list-reversing lemma, the author could then write:
\begin{minted}[frame=none]{haskell}
newtype Reverse xs = Reverse Defn

gdpRev :: ([a] ~~ xs) -> ([a] ~~ Reverse xs)
gdpRev xs = defn (reverse (the xs))

\end{minted}

\begin{minted}[frame=none]{haskell}
\end{minted}

\begin{filecontents*}{pred.hs}
-- Type exported, constructor hidden (but see `axiom`)
data Proof p = QED deriving (Functor, Applicative, Monad)

-- Attaching predicates to values
newtype a ::: p = SuchThat Defn

(...) :: a -> Proof p -> (a ::: p)
x ...proof = coerce x

-- Logical constants. We can use empty data declarations,
-- because these types are only used as phantoms.
data TRUE
data FALSE
data p && q
data p || q
data p --> q
data Not p
data p == q

-- Natural deduction rules (implementations all
-- ignore parameters and return `QED`)
and_intro   :: p     ->   q       -> Proof (p && q)
or_elimL    :: (p || q)           -> Proof p
impl_intro  :: (p -> Proof q)     -> Proof (p --> q)
impl_elim   :: (p --> q)   ->  p  -> Proof q
not_intro   :: (p -> Proof FALSE) -> Proof (Not p)
contradicts :: p     ->   Not p   -> Proof FALSE
absurd      :: FALSE              -> Proof p
refl        ::                       Proof (x == x)
  -- ... and many more

-- Exported function that allows library authors to
-- assert arbitrary axioms about their API.
axiom :: Proof p
axiom = QED
\end{filecontents*}

\begin{figure}
  \inputminted{haskell}{pred.hs}
  \caption{Basic constants and functions for building up the ``proofs''
    in ``ghosts of departed proofs''. 
    \label{predicate-logic}}
\end{figure}

\subsection{Building theory libraries}
In both the \texttt{St} and the \texttt{justified-containers} case studies,
the library author exported proof combinators that encoded basic facts about
the algebra of sets. Such redundancy is undesirable for the library authors,
who have to do work, and also for the end user who has to remember dozens of
variations on the same basic proof combinators.

Luckily, it is simple to separate axioms and inference laws for specific theories
from the libraries that make use of them. For example, we could publish a small
library containing basic predicates and deduction rules about sets, such as:
\begin{minted}[frame=none]{haskell}
newtype x #$\in$# xs  = Element   Defn
newtype xs #$\subseteq$# ys = Subset    Defn
newtype xs #$\cap$# ys = Intersect Defn

subset_elts :: (a #$\subseteq$# b) -> (x #$\in$# a) -> Proof (x #$\in$# b)
subset_elts _ _ = axiom

intersect_subset :: Proof ((a #$\cap$# b) #$\subseteq$# a)
intersect_subset = axiom

intersect_sym :: Proof (a #$\cap$# b == b #$\cap$# a)
intersect_sym = axiom
\end{minted}
This same theory library could be used to reason about shared \texttt{St}
regions and about key sets for maps, already achieving code reuse within
the narrow confines of this paper's examples.

\subsection{Ghosts on the outside, proofs on the inside}
Factoring common lemmas into theory modules helps ensure that
the module exports give a sound model of the structure they are
describing. The author only needs to check the validity of their own
theory module; others can then re-use that validation for free.

But how can the author of a theory library be confident that they wrote down
the correct lemmas in the first place? There is always the ``think very hard
about it'' approach, but we can do better. Formal verification tools like
Liquid Haskell \cite{vazou2016liquid} or \texttt{hs-to-coq} \cite{spector2018total}
can be applied by the author of the theory library to ensure that it presents a
sound interface. Note that the tooling is only needed by the author of the library;
the end-users of the theory library only need to use plain Haskell.

An extra level of confidence can be obtained by splitting the theory library
into \texttt{Assumed} and \texttt{Derived} submodules. A small, core set of
axioms are placed in the \texttt{Assumed} module and possibly checked by
external tools. A larger set of lemmas, derived from these axioms using GDP
proof combinators, resides in the \texttt{Derived} submodule. This separation
allows the library author to build up a large collection of lemmas from a
small---and hopefully easy-to-verify---set of basic axioms.

\subsection{Building custom proof tactics}\label{tactics}

For simple properties, the task of writing a proof is not too difficult. But for
more sophisticated properties, the deployment of \emph{proof tactics} becomes
critical. A proof tactic is a search strategy for proofs, usually targeted at
proving one particular class of theorems. For example, the \texttt{Coq} tactic
\texttt{omega} is useful for proving theorems about arithmetic, while
\texttt{simpl} is useful for simplifying a complex goal.

Tactics are often designed with a specific domain in mind; to be most useful,
theory creators (and library authors) should be able to create their own tactics
when needed. For example, a library dealing extensively with fixed-width numeric types
may benefit from specific proof tactics based around the theory of bitvectors.

One approach to providing custom tactics is to leverage \texttt{GHC}'s support for
\emph{type-checker plugins}. These plugins hook into \texttt{GHC}'s $\textsf{OutsideIn}(X)$
inference algorithm \cite{vytiniotis2011outsidein}, teaching the algorithm how to solve
new kinds of type constraints.

As a proof-of-concept, we developed a simple typechecker plugin that implements
proof by analytic tableaux \cite{smullyan1995first}. This tactic can verify the satisfiability of any
valid formula of propositional logic; the na\"ive implementation takes about
60 lines of Haskell, plus 150 lines of glue code to mediate between
the tableaux solver and $\textsf{OutsideIn}(X)$.

To trigger the custom tactic, we introduce an empty injective type family---hidden
from the user---and a single exported function \texttt{tableaux}.
\begin{minted}[frame=none]{haskell}
type family ProofByTableaux p = p' | p' -> p

tableaux :: ProofByTableaux p
tableaux = error "proof by analytic tableaux."
\end{minted}
Morally, we want to think of \texttt{ProofByTableaux p} as an alias for \texttt{p}.
The trick is that our plugin will first check that the proposition \texttt{p} is
a valid formula of predicate logic. Only then will the plugin allow \texttt{GHC}
to replace \texttt{ProofByTableaux p} with \texttt{p}.

Our type-checker plugin will get a chance to intervene whenever \texttt{GHC}
meets a type equality constraint of the form \texttt{ProofByTableaux p $\sim$ p'}.
When such a constraint is met, the plugin performs the following tasks:
\begin{enumerate}
\item Convert the \emph{type} \texttt{p'} to a \emph{formula} $\Phi$ in propositional
  logic, introducing free variables for any subterms that are not built from the
  propositional logic type constructors \verb|&&|, \verb#||#, \verb|-->|, and \verb|Not|.
\item Invoke the solver for analytic tableaux, attempting to find an assignment of truth
  values to variables such that $\neg \Phi$ is true.
\item If the solver finds such a truth assignment, report an error: the user asked us
  to apply the tactic to prove $\Phi$, but the variable assignment demonstrated that
  $\Phi$ is actually \emph{false}.
\item If the solver finds that no such truth assignment exists, then we have proven that
  $\Phi$ is valid. Tell \texttt{GHC} to discharge the constraint \texttt{ProofByTableaux p $\sim$ p'}, replacing it with \texttt{p $\sim$ p'}.
\end{enumerate}
For the user, the effect appears to be that \texttt{tableaux} can act as a value of
type \texttt{Proof p} whenever \texttt{p} is a valid formula in propositional logic.
A glance at \Cref{tableaux-example} demonstrates why proof tactics are so desirable:
the user can just wave their hands and say ``this is true by basic facts from predicate logic,''
instead of constructing a proof by hand.

\section*{Summary}
****

\bibliographystyle{abbrvnat}

\bibliography{gdp.bib}

\end{document}
